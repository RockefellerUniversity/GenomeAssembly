% Options for packages loaded elsewhere
\PassOptionsToPackage{unicode}{hyperref}
\PassOptionsToPackage{hyphens}{url}
%
\documentclass[
]{article}
\usepackage{lmodern}
\usepackage{amssymb,amsmath}
\usepackage{ifxetex,ifluatex}
\ifnum 0\ifxetex 1\fi\ifluatex 1\fi=0 % if pdftex
  \usepackage[T1]{fontenc}
  \usepackage[utf8]{inputenc}
  \usepackage{textcomp} % provide euro and other symbols
\else % if luatex or xetex
  \usepackage{unicode-math}
  \defaultfontfeatures{Scale=MatchLowercase}
  \defaultfontfeatures[\rmfamily]{Ligatures=TeX,Scale=1}
\fi
% Use upquote if available, for straight quotes in verbatim environments
\IfFileExists{upquote.sty}{\usepackage{upquote}}{}
\IfFileExists{microtype.sty}{% use microtype if available
  \usepackage[]{microtype}
  \UseMicrotypeSet[protrusion]{basicmath} % disable protrusion for tt fonts
}{}
\makeatletter
\@ifundefined{KOMAClassName}{% if non-KOMA class
  \IfFileExists{parskip.sty}{%
    \usepackage{parskip}
  }{% else
    \setlength{\parindent}{0pt}
    \setlength{\parskip}{6pt plus 2pt minus 1pt}}
}{% if KOMA class
  \KOMAoptions{parskip=half}}
\makeatother
\usepackage{xcolor}
\IfFileExists{xurl.sty}{\usepackage{xurl}}{} % add URL line breaks if available
\IfFileExists{bookmark.sty}{\usepackage{bookmark}}{\usepackage{hyperref}}
\hypersetup{
  pdfauthor={Rockefeller University, The Vertebrate Genome Lab},
  hidelinks,
  pdfcreator={LaTeX via pandoc}}
\urlstyle{same} % disable monospaced font for URLs
\usepackage[margin=1in]{geometry}
\usepackage{color}
\usepackage{fancyvrb}
\newcommand{\VerbBar}{|}
\newcommand{\VERB}{\Verb[commandchars=\\\{\}]}
\DefineVerbatimEnvironment{Highlighting}{Verbatim}{commandchars=\\\{\}}
% Add ',fontsize=\small' for more characters per line
\usepackage{framed}
\definecolor{shadecolor}{RGB}{248,248,248}
\newenvironment{Shaded}{\begin{snugshade}}{\end{snugshade}}
\newcommand{\AlertTok}[1]{\textcolor[rgb]{0.94,0.16,0.16}{#1}}
\newcommand{\AnnotationTok}[1]{\textcolor[rgb]{0.56,0.35,0.01}{\textbf{\textit{#1}}}}
\newcommand{\AttributeTok}[1]{\textcolor[rgb]{0.77,0.63,0.00}{#1}}
\newcommand{\BaseNTok}[1]{\textcolor[rgb]{0.00,0.00,0.81}{#1}}
\newcommand{\BuiltInTok}[1]{#1}
\newcommand{\CharTok}[1]{\textcolor[rgb]{0.31,0.60,0.02}{#1}}
\newcommand{\CommentTok}[1]{\textcolor[rgb]{0.56,0.35,0.01}{\textit{#1}}}
\newcommand{\CommentVarTok}[1]{\textcolor[rgb]{0.56,0.35,0.01}{\textbf{\textit{#1}}}}
\newcommand{\ConstantTok}[1]{\textcolor[rgb]{0.00,0.00,0.00}{#1}}
\newcommand{\ControlFlowTok}[1]{\textcolor[rgb]{0.13,0.29,0.53}{\textbf{#1}}}
\newcommand{\DataTypeTok}[1]{\textcolor[rgb]{0.13,0.29,0.53}{#1}}
\newcommand{\DecValTok}[1]{\textcolor[rgb]{0.00,0.00,0.81}{#1}}
\newcommand{\DocumentationTok}[1]{\textcolor[rgb]{0.56,0.35,0.01}{\textbf{\textit{#1}}}}
\newcommand{\ErrorTok}[1]{\textcolor[rgb]{0.64,0.00,0.00}{\textbf{#1}}}
\newcommand{\ExtensionTok}[1]{#1}
\newcommand{\FloatTok}[1]{\textcolor[rgb]{0.00,0.00,0.81}{#1}}
\newcommand{\FunctionTok}[1]{\textcolor[rgb]{0.00,0.00,0.00}{#1}}
\newcommand{\ImportTok}[1]{#1}
\newcommand{\InformationTok}[1]{\textcolor[rgb]{0.56,0.35,0.01}{\textbf{\textit{#1}}}}
\newcommand{\KeywordTok}[1]{\textcolor[rgb]{0.13,0.29,0.53}{\textbf{#1}}}
\newcommand{\NormalTok}[1]{#1}
\newcommand{\OperatorTok}[1]{\textcolor[rgb]{0.81,0.36,0.00}{\textbf{#1}}}
\newcommand{\OtherTok}[1]{\textcolor[rgb]{0.56,0.35,0.01}{#1}}
\newcommand{\PreprocessorTok}[1]{\textcolor[rgb]{0.56,0.35,0.01}{\textit{#1}}}
\newcommand{\RegionMarkerTok}[1]{#1}
\newcommand{\SpecialCharTok}[1]{\textcolor[rgb]{0.00,0.00,0.00}{#1}}
\newcommand{\SpecialStringTok}[1]{\textcolor[rgb]{0.31,0.60,0.02}{#1}}
\newcommand{\StringTok}[1]{\textcolor[rgb]{0.31,0.60,0.02}{#1}}
\newcommand{\VariableTok}[1]{\textcolor[rgb]{0.00,0.00,0.00}{#1}}
\newcommand{\VerbatimStringTok}[1]{\textcolor[rgb]{0.31,0.60,0.02}{#1}}
\newcommand{\WarningTok}[1]{\textcolor[rgb]{0.56,0.35,0.01}{\textbf{\textit{#1}}}}
\usepackage{longtable,booktabs}
% Correct order of tables after \paragraph or \subparagraph
\usepackage{etoolbox}
\makeatletter
\patchcmd\longtable{\par}{\if@noskipsec\mbox{}\fi\par}{}{}
\makeatother
% Allow footnotes in longtable head/foot
\IfFileExists{footnotehyper.sty}{\usepackage{footnotehyper}}{\usepackage{footnote}}
\makesavenoteenv{longtable}
\usepackage{graphicx,grffile}
\makeatletter
\def\maxwidth{\ifdim\Gin@nat@width>\linewidth\linewidth\else\Gin@nat@width\fi}
\def\maxheight{\ifdim\Gin@nat@height>\textheight\textheight\else\Gin@nat@height\fi}
\makeatother
% Scale images if necessary, so that they will not overflow the page
% margins by default, and it is still possible to overwrite the defaults
% using explicit options in \includegraphics[width, height, ...]{}
\setkeys{Gin}{width=\maxwidth,height=\maxheight,keepaspectratio}
% Set default figure placement to htbp
\makeatletter
\def\fps@figure{htbp}
\makeatother
\setlength{\emergencystretch}{3em} % prevent overfull lines
\providecommand{\tightlist}{%
  \setlength{\itemsep}{0pt}\setlength{\parskip}{0pt}}
\setcounter{secnumdepth}{-\maxdimen} % remove section numbering

\title{RU Genome Assembly, Session 1}
\author{Rockefeller University, The Vertebrate Genome Lab}
\date{}

\begin{document}
\maketitle

{
\setcounter{tocdepth}{2}
\tableofcontents
}
\hypertarget{genome-assembly-session1}{%
\section{Genome Assembly (session1)}\label{genome-assembly-session1}}

\begin{center}\rule{0.5\linewidth}{0.5pt}\end{center}

\hypertarget{overview}{%
\subsection{Overview}\label{overview}}

??We often put an overview slide that links to the various parts of the
course content to make it easy to navigate i.e.~this one for intro to
r??

\begin{itemize}
\tightlist
\item
  \href{https://rockefelleruniversity.github.io/Intro_To_R_1Day/r_course/presentations/singlepage/introToR_Session1.html\#set-up}{Set
  up}
\item
  \href{https://rockefelleruniversity.github.io/Intro_To_R_1Day/r_course/presentations/singlepage/introToR_Session1.html\#background-to-r}{Background
  to R}
\item
  \href{https://rockefelleruniversity.github.io/Intro_To_R_1Day/r_course/presentations/singlepage/introToR_Session1.html\#data_types_in_r}{Data
  types in R}
\item
  \href{https://rockefelleruniversity.github.io/Intro_To_R_1Day/r_course/presentations/singlepage/introToR_Session1.html\#reading-and-writing-data-in-r}{Reading
  and writing in R}
\end{itemize}

\hypertarget{materials}{%
\subsection{Materials}\label{materials}}

??We try to denote each major section of content with these title slides
(see above for set up). The hierarchy will help with the different
versions of the teaching content i.e.~slides and single page.??

?? Materials just gives the links to download the package Just ensure
links are up to date??

All prerequisites, links to material and slides for this course can be
found on github.

\begin{itemize}
\tightlist
\item
  \href{https://rockefelleruniversity.github.io/Intro_To_R_1Day/}{Intro\_To\_R\_1}
\end{itemize}

Or can be downloaded as a zip archive from here.

\begin{itemize}
\tightlist
\item
  \href{https://github.com/rockefelleruniversity/Intro_To_R_1Day/zipball/master}{Download
  zip}
\end{itemize}

\hypertarget{set-the-working-directory}{%
\subsection{Set the Working directory}\label{set-the-working-directory}}

Before running any of the code in the practicals or slides we need to
set the working directory to the folder we unarchived.

You may navigate to the unarchived RU\_Course\_help folder in the
Rstudio menu.

\textbf{Session -\textgreater{} Set Working Directory -\textgreater{}
Choose Directory}

or in the console.

\begin{Shaded}
\begin{Highlighting}[]
\KeywordTok{setwd}\NormalTok{(}\StringTok{"/PathToMyDownload/RU_Course_template/r_course"}\NormalTok{)}
\CommentTok{# e.g. setwd('~/Downloads/Intro_To_R_1Day/r_course')}
\end{Highlighting}
\end{Shaded}

\hypertarget{slide-structure}{%
\subsection{Slide Structure}\label{slide-structure}}

We use the a double has/pound for slides. This is important for
hierarchy. If you use 1 it will work for the slides, but will mess up
the single page version.

\hypertarget{leftright-alignment}{%
\subsection{Left/Right alignment}\label{leftright-alignment}}

To show things side by side on a slide you can use pull left/right. t is
important to keep the tab on the second square bracket. On single page
it will just be one after the other:

.pull-left{[} This will be on left of slide {]}

.pull-right{[} This will be on right{]}

\hypertarget{data-from-external-sources}{%
\subsection{Data from external
sources}\label{data-from-external-sources}}

everything in the rmd is relative to the inst/extdata directory so you
have some data to load in, just put it in the data directory, read it in
along the path. The structure is maintained for those attending the
course and download the package using the instructions. The same rules
are followed for imgs and scripts that you may call.

\begin{Shaded}
\begin{Highlighting}[]
\NormalTok{Table <-}\StringTok{ }\KeywordTok{read.table}\NormalTok{(}\StringTok{"../data/readThisTable.csv"}\NormalTok{, }\DataTypeTok{sep =} \StringTok{","}\NormalTok{, }\DataTypeTok{header =}\NormalTok{ T)}
\end{Highlighting}
\end{Shaded}

\hypertarget{big-processes}{%
\subsection{Big Processes}\label{big-processes}}

If what you want to do is super intense computationally you can use the
eval=F arguments to prevent it being done during compilation.

\begin{Shaded}
\begin{Highlighting}[]
\CommentTok{# Intense computation}
\NormalTok{myresult <-}\StringTok{ }\DecValTok{10}\OperatorTok{^}\DecValTok{6} \OperatorTok{+}\StringTok{ }\DecValTok{1}
\end{Highlighting}
\end{Shaded}

You can then just load in the result of the computation, and keep using
the result after that. If you do want them to evaluate the intense
computation, then you can hide this intermediate file that you are
loading in with echo=F.

\begin{longtable}[]{@{}l@{}}
\toprule
\endhead
\begin{minipage}[t]{0.04\columnwidth}\raggedright
\#\# Exercises The following few slides show you how to structure
exercise slides.\strut
\end{minipage}\tabularnewline
\begin{minipage}[t]{0.04\columnwidth}\raggedright
We often have several exercise slides per session. So you can just copy
and paste and change the directory to the appropriate name. All 3 file
types are made from you single exercise Rmd.\strut
\end{minipage}\tabularnewline
\bottomrule
\end{longtable}

\hypertarget{time-for-an-exercise}{%
\subsection{Time for an exercise!}\label{time-for-an-exercise}}

??? exercise description here??
\href{../../exercises/exercises/MyExercise1_exercises.html}{here}

\hypertarget{time-for-an-exercise-1}{%
\subsection{Time for an exercise!}\label{time-for-an-exercise-1}}

??? exercise description here??
\href{../../exercises/exercises/MyExercise2_exercises.html}{here}

\begin{center}\rule{0.5\linewidth}{0.5pt}\end{center}

\hypertarget{answers-to-exercise}{%
\subsection{Answers to exercise}\label{answers-to-exercise}}

Answers can be found
\href{../../exercises/answers/MyExercise2_answers.html}{here}

R code for solutions can be found
\href{../../exercises/answers/MyExercise2_answers.R}{here}

\end{document}
